\documentclass[12pt, a4paper]{article}\usepackage[]{graphicx}\usepackage[]{color}
%% maxwidth is the original width if it is less than linewidth
%% otherwise use linewidth (to make sure the graphics do not exceed the margin)
\makeatletter
\def\maxwidth{ %
  \ifdim\Gin@nat@width>\linewidth
    \linewidth
  \else
    \Gin@nat@width
  \fi
}
\makeatother

\definecolor{fgcolor}{rgb}{0.345, 0.345, 0.345}
\newcommand{\hlnum}[1]{\textcolor[rgb]{0.686,0.059,0.569}{#1}}%
\newcommand{\hlstr}[1]{\textcolor[rgb]{0.192,0.494,0.8}{#1}}%
\newcommand{\hlcom}[1]{\textcolor[rgb]{0.678,0.584,0.686}{\textit{#1}}}%
\newcommand{\hlopt}[1]{\textcolor[rgb]{0,0,0}{#1}}%
\newcommand{\hlstd}[1]{\textcolor[rgb]{0.345,0.345,0.345}{#1}}%
\newcommand{\hlkwa}[1]{\textcolor[rgb]{0.161,0.373,0.58}{\textbf{#1}}}%
\newcommand{\hlkwb}[1]{\textcolor[rgb]{0.69,0.353,0.396}{#1}}%
\newcommand{\hlkwc}[1]{\textcolor[rgb]{0.333,0.667,0.333}{#1}}%
\newcommand{\hlkwd}[1]{\textcolor[rgb]{0.737,0.353,0.396}{\textbf{#1}}}%
\let\hlipl\hlkwb

\usepackage{framed}
\makeatletter
\newenvironment{kframe}{%
 \def\at@end@of@kframe{}%
 \ifinner\ifhmode%
  \def\at@end@of@kframe{\end{minipage}}%
  \begin{minipage}{\columnwidth}%
 \fi\fi%
 \def\FrameCommand##1{\hskip\@totalleftmargin \hskip-\fboxsep
 \colorbox{shadecolor}{##1}\hskip-\fboxsep
     % There is no \\@totalrightmargin, so:
     \hskip-\linewidth \hskip-\@totalleftmargin \hskip\columnwidth}%
 \MakeFramed {\advance\hsize-\width
   \@totalleftmargin\z@ \linewidth\hsize
   \@setminipage}}%
 {\par\unskip\endMakeFramed%
 \at@end@of@kframe}
\makeatother

\definecolor{shadecolor}{rgb}{.97, .97, .97}
\definecolor{messagecolor}{rgb}{0, 0, 0}
\definecolor{warningcolor}{rgb}{1, 0, 1}
\definecolor{errorcolor}{rgb}{1, 0, 0}
\newenvironment{knitrout}{}{} % an empty environment to be redefined in TeX

\usepackage{alltt}

%%%%%%%%%%%%%%%%%%%%%%%%%%%%%%%%%%%%%%%%%%%%%%%%%%%%%%%%%%%%%%%%
% LaTeX packages
%\usepackage[OT4]{polski}
\usepackage[utf8]{inputenc}
\usepackage[top=2.5cm, bottom=2.5cm, left=2cm, right=2cm]{geometry}
\usepackage{graphicx}
\usepackage{amsmath}
\usepackage{float}
\usepackage[colorlinks=true, linkcolor=blue]{hyperref}


%%%%%%%%%%%%%%%%%%%%%%%%%%%%%%%%%%%%%%%%%%%%%%%%%%%%%%%%%%%%%%%%
% global settings


\IfFileExists{upquote.sty}{\usepackage{upquote}}{}
\begin{document}

%%%%%%%%%%%%%%%%%%%%%%%%%%%%%%%%%%%%%%%%%%%%%%%%%%%%%%%%%%%%%%%%
% title page
\title{Estimation theory -- Report 3}
\author{Marta Frankowska, 208581 \\ Agnieszka Szkutek, 208619}
\maketitle
\tableofcontents 


%%%%%%%%%%%%%%%%%%%%%%%%%%%%%%%%%%%%%%%%%%%%%%%%%%%%%%%%%%%%%%%%
\section{Model}
In both exercises we will be using the Factor model
\[ Y_{T\times N} = F_{T\times K} \cdot \lambda_{K\times N} + e_{T\times N},\]
where 
\begin{itemize}
  \item $Y_{T\times N}$ panel of observations
  \item $F_{T\times K}$ matrix of common (latent) factors
  \item $\lambda_{K\times N}$ matrix of loadings
  \item $e_{T\times N}$ panel of specific components
\end{itemize}

To calculate $F$ and $\lambda$ we use the following formulas:
\[ \hat{F} = \sqrt{T} V_{1:K} \quad\text{and}\quad \hat{\lambda} = \frac{\hat{F}' Y}{T},\]
where $V_{1:K}$ are eigenvectors of $YY'$ corresponding to the $K$ largest eigenvalues.


\subsection{Selecting optimal number of factors}
Notation:
\begin{itemize}
  \item $K = 1,2,\dots,K_\text{max}$ - the number of factors,
  \item $e^{(K)}$ - the individual components for $K$ factors,
  \item $V(K) = \frac{1}{NT}\sum_{t=1}^{T} \sum_{i=1}^{N} \left( e^{(K)}_{t i} \right)^2$,
  \item $\hat{\sigma}^2 = V(K_\text{max})$ - consistent estimator of variance.
\end{itemize}
Information criteria:
\begin{itemize}
  \item $PC_1(K) = V(K) + K \hat{\sigma}^2 \frac{N+T}{NT} \ln{\frac{NT}{N+T}}$,
  \item $IPC_1(K) = \log{V(K)} + K\frac{N+T}{NT} \ln{\frac{NT}{N+T}}$.
\end{itemize}
Algorithm:
\begin{enumerate}
	\item Set $K_\text{max}$;
	\item Compute $IC(K)$ for $K = 1,\dots,K_\text{max}$;
	\item Choose $\hat{K}$ such that $IC(\hat{K}) = \min_{1\leq K\leq K_\text{max}} IC(K)$.
\end{enumerate}


\begin{knitrout}
\definecolor{shadecolor}{rgb}{0.969, 0.969, 0.969}\color{fgcolor}\begin{kframe}
\begin{alltt}
\hlstd{factor.model.est} \hlkwb{<-}
  \hlkwa{function}\hlstd{(}\hlkwc{Y}\hlstd{,} \hlkwc{K_max}\hlstd{,} \hlkwc{draw}\hlstd{)} \hlcom{#function returning which K we should choose}
  \hlstd{\{}
    \hlstd{T} \hlkwb{<-} \hlkwd{nrow}\hlstd{(Y)}
    \hlstd{N} \hlkwb{<-} \hlkwd{ncol}\hlstd{(Y)}

    \hlstd{eigen.decomp} \hlkwb{<-} \hlkwd{eigen}\hlstd{(Y} \hlopt \hlkwd{t}\hlstd{(Y))}
    \hlstd{eigen.values} \hlkwb{<-} \hlstd{eigen.decomp}\hlopt{$}\hlstd{values}
    \hlstd{eigen.vectors} \hlkwb{<-} \hlstd{eigen.decomp}\hlopt{$}\hlstd{vector}
    \hlcom{# we calculate F, lambda and e for K_max}
    \hlstd{F} \hlkwb{<-} \hlkwd{sqrt}\hlstd{(T)} \hlopt{*} \hlstd{eigen.vectors[,} \hlnum{1}\hlopt{:}\hlstd{K_max]}
    \hlstd{lambda} \hlkwb{<-} \hlkwd{t}\hlstd{(F)} \hlopt \hlstd{Y} \hlopt{/} \hlstd{T}
    \hlstd{e} \hlkwb{<-} \hlstd{Y} \hlopt{-} \hlstd{F} \hlopt \hlstd{lambda}
    \hlstd{sigma2.hat} \hlkwb{<-} \hlkwd{sum}\hlstd{(e} \hlopt{^} \hlnum{2}\hlstd{)} \hlopt{/} \hlstd{(N} \hlopt{*} \hlstd{T)}

    \hlstd{PC1} \hlkwb{<-} \hlnum{1}\hlopt{:}\hlstd{K_max}
    \hlstd{IPC1} \hlkwb{<-} \hlnum{1}\hlopt{:}\hlstd{K_max}
    \hlkwa{for} \hlstd{(K} \hlkwa{in} \hlnum{1}\hlopt{:}\hlstd{K_max)}
    \hlstd{\{}
      \hlcom{# we calculate F, lambda and e for K}
      \hlstd{F} \hlkwb{<-} \hlkwd{sqrt}\hlstd{(T)} \hlopt{*} \hlstd{eigen.vectors[,} \hlnum{1}\hlopt{:}\hlstd{K]}
      \hlstd{lambda} \hlkwb{<-} \hlkwd{t}\hlstd{(F)} \hlopt \hlstd{Y} \hlopt{/} \hlstd{T}
      \hlstd{e} \hlkwb{<-} \hlstd{Y} \hlopt{-} \hlstd{F} \hlopt \hlstd{lambda}
      \hlstd{V} \hlkwb{<-} \hlkwd{sum}\hlstd{(e} \hlopt{^} \hlnum{2}\hlstd{)} \hlopt{/} \hlstd{(N} \hlopt{*} \hlstd{T)}
      \hlcom{# we calculate PC1 and IPC1 for K}
      \hlstd{PC1[K]} \hlkwb{<-}
        \hlstd{V} \hlopt{+} \hlstd{K} \hlopt{*} \hlstd{sigma2.hat} \hlopt{*} \hlstd{((N} \hlopt{+} \hlstd{T)} \hlopt{/} \hlstd{(N} \hlopt{*} \hlstd{T))} \hlopt{*} \hlkwd{log}\hlstd{(N} \hlopt{*} \hlstd{T} \hlopt{/} \hlstd{(N} \hlopt{+} \hlstd{T))}
      \hlstd{IPC1[K]} \hlkwb{<-}
        \hlkwd{log}\hlstd{(V)} \hlopt{+} \hlstd{K} \hlopt{*} \hlstd{((N} \hlopt{+} \hlstd{T)} \hlopt{/} \hlstd{(N} \hlopt{*} \hlstd{T))} \hlopt{*} \hlkwd{log}\hlstd{(N} \hlopt{*} \hlstd{T} \hlopt{/} \hlstd{(N} \hlopt{+} \hlstd{T))}
    \hlstd{\}}
    \hlcom{# choose K which gives the minimal value}
    \hlstd{PC1_K} \hlkwb{<-} \hlkwd{which.min}\hlstd{(PC1)}
    \hlstd{IPC1_K} \hlkwb{<-} \hlkwd{which.min}\hlstd{(IPC1)}

    \hlkwa{if} \hlstd{(draw) \{}
      \hlstd{max.y} \hlkwb{<-} \hlkwd{max}\hlstd{(}\hlkwd{max}\hlstd{(IPC1),} \hlkwd{max}\hlstd{(PC1))} \hlopt{+} \hlnum{2}
      \hlkwd{par}\hlstd{(}\hlkwc{mfrow} \hlstd{=} \hlkwd{c}\hlstd{(}\hlnum{1}\hlstd{,}\hlnum{1}\hlstd{),} \hlkwc{mar}\hlstd{=}\hlkwd{c}\hlstd{(}\hlnum{4}\hlstd{,}\hlnum{4}\hlstd{,}\hlnum{1}\hlstd{,}\hlnum{2}\hlstd{))}
      \hlkwd{matplot}\hlstd{(}\hlnum{1}\hlopt{:}\hlstd{K_max,} \hlkwd{cbind}\hlstd{(IPC1, PC1),} \hlkwc{pch}\hlstd{=}\hlnum{1}\hlstd{,} \hlkwc{col}\hlstd{=}\hlkwd{c}\hlstd{(}\hlstr{"blue"}\hlstd{,} \hlstr{"red"}\hlstd{),}
              \hlkwc{xlab}\hlstd{=}\hlstr{"K"}\hlstd{,} \hlkwc{ylab}\hlstd{=}\hlstr{"IC"}\hlstd{,}
              \hlkwc{ylim} \hlstd{=} \hlkwd{c}\hlstd{(}\hlkwd{min}\hlstd{(PC1[PC1_K],IPC1[IPC1_K]),}
                       \hlkwd{max}\hlstd{(}\hlkwd{max}\hlstd{(IPC1),}\hlkwd{max}\hlstd{(PC1))}\hlopt{+}\hlnum{2}\hlstd{))}
      \hlkwd{legend}\hlstd{(}\hlnum{1}\hlstd{, max.y,} \hlkwd{c}\hlstd{(}\hlstr{"IPC1"}\hlstd{,} \hlstr{"PC1"}\hlstd{),} \hlkwc{col} \hlstd{=} \hlkwd{c}\hlstd{(}\hlstr{"blue"}\hlstd{,} \hlstr{"red"}\hlstd{),} \hlkwc{pch}\hlstd{=}\hlnum{1}\hlstd{)}
    \hlstd{\}}
    \hlkwd{return} \hlstd{(}\hlkwd{list}\hlstd{(PC1_K, IPC1_K))}
  \hlstd{\}}
\end{alltt}
\end{kframe}
\end{knitrout}

\section{Exercise 1}
We will be using data from file \textit{dataLab3.xlsx}, where $Y$ size is $T\times N = 100\times 100$.
To calculate the number of factors $K$ we will use function \textit{factor.model.est}.


\subsection{Part 1}
First, we calculate the number of factors for the whole sample. 
\begin{knitrout}
\definecolor{shadecolor}{rgb}{0.969, 0.969, 0.969}\color{fgcolor}\begin{figure}[H]

{\centering \includegraphics[width=\maxwidth]{figure/ex1_1_IC-1} 

}

\caption[Information criteria]{Information criteria}\label{fig:ex1.1.IC}
\end{figure}


\end{knitrout}
The function returns the same $\hat{K}$ for both $PC_1$ and $IPC_1$, it is equal to 
% latex table generated in R 3.2.3 by xtable 1.8-2 package
% Fri Dec  1 12:54:26 2017
\begin{table}[H]
\centering
\begin{tabular}{rrr}
  \hline
  & PC1 & IPC1 \\ 
  \hline
K & 3.00 & 3.00 \\ 
   \hline
\end{tabular}
\caption{K returned for both Information criteria} 
\label{tab:K}
\end{table}




Share of explained variance
\begin{equation}
\label{share}
\frac{\sum_{i=1}^{K} \gamma_i }{\sum_{i=1}^{T} \gamma_i},
\end{equation}
where $\gamma_i$ are the eigenvalues of $Y Y'$ can also be used to choose the number of factors.


The plot shows the variability of the factors
\begin{knitrout}
\definecolor{shadecolor}{rgb}{0.969, 0.969, 0.969}\color{fgcolor}\begin{figure}[H]

{\centering \includegraphics[width=\maxwidth]{figure/ex1_1-1} 

}

\caption[Share of explained variance depending on K]{Share of explained variance depending on K}\label{fig:ex1.1}
\end{figure}


\end{knitrout}

We can observe that for $K$ we calculated from Information Criteria, the share of explained variance is more than 0.8, and to be exact we can calculate it from the formula \eqref{share}:
\begin{knitrout}
\definecolor{shadecolor}{rgb}{0.969, 0.969, 0.969}\color{fgcolor}\begin{kframe}
\begin{verbatim}
## [1] 0.8437638
\end{verbatim}
\end{kframe}
\end{knitrout}


We can also say that the results are correct by looking at the eigenvalues
\begin{knitrout}
\definecolor{shadecolor}{rgb}{0.969, 0.969, 0.969}\color{fgcolor}\begin{figure}[H]

{\centering \includegraphics[width=\maxwidth]{figure/ex1_eigenval-1} 

}

\caption[Eigenvalues of YY']{Eigenvalues of YY'}\label{fig:ex1.eigenval}
\end{figure}


\end{knitrout}



Now we can calculate $\hat{F}$ and $\hat{\lambda}$
\begin{knitrout}
\definecolor{shadecolor}{rgb}{0.969, 0.969, 0.969}\color{fgcolor}\begin{kframe}
\begin{alltt}
\hlkwd{source}\hlstd{(}\hlstr{"functions.R"}\hlstd{)}
\hlstd{Y} \hlkwb{<-} \hlkwd{as.matrix}\hlstd{(}\hlkwd{read_excel}\hlstd{(}\hlstr{'dataLab3.xlsx'}\hlstd{,} \hlkwc{col_names} \hlstd{=} \hlnum{FALSE}\hlstd{))}
\hlstd{N} \hlkwb{<-} \hlkwd{ncol}\hlstd{(Y); T} \hlkwb{<-} \hlkwd{nrow}\hlstd{(Y)}
\hlstd{K} \hlkwb{<-} \hlkwd{as.numeric}\hlstd{(}\hlkwd{factor.model.est}\hlstd{(Y,} \hlnum{10}\hlstd{,} \hlnum{FALSE}\hlstd{)[}\hlnum{1}\hlstd{])}

\hlstd{eigen.decomp} \hlkwb{<-} \hlkwd{eigen}\hlstd{(Y} \hlopt \hlkwd{t}\hlstd{(Y))}
\hlstd{eigen.vectors} \hlkwb{<-} \hlstd{eigen.decomp}\hlopt{$}\hlstd{vectors}
\hlstd{F} \hlkwb{<-} \hlkwd{sqrt}\hlstd{(T)} \hlopt{*} \hlstd{eigen.vectors[,} \hlnum{1}\hlopt{:}\hlstd{K]}
\hlstd{lambda} \hlkwb{<-} \hlkwd{t}\hlstd{(F)} \hlopt \hlstd{Y} \hlopt{/} \hlstd{T}
\end{alltt}
\end{kframe}
\end{knitrout}
To check if the results are correct, we can check the following condition
\[ \frac{F'F}{T}=I \]

We calculate $\frac{F'F}{T}$ and the result is as follows
\begin{knitrout}
\definecolor{shadecolor}{rgb}{0.969, 0.969, 0.969}\color{fgcolor}\begin{kframe}
\begin{verbatim}
##               [,1]         [,2]          [,3]
## [1,]  1.000000e+00 3.774758e-17 -3.474998e-16
## [2,]  3.774758e-17 1.000000e+00  2.642331e-16
## [3,] -3.474998e-16 2.642331e-16  1.000000e+00
\end{verbatim}
\end{kframe}
\end{knitrout}
Taking into account the numerical errors we can assume that the above matrix is an identity matrix.




\subsection{Part 2 and 3}

Now we will compare estimated number of factors for the whole sample, the first 20 columns and the first 20 rows.
% latex table generated in R 3.2.3 by xtable 1.8-2 package
% Fri Dec  1 12:54:26 2017
\begin{table}[H]
\centering
\begin{tabular}{rrr}
  \hline
  & PC1 & IPC1 \\ 
  \hline
whole sample & 3.00 & 3.00 \\ 
  first 20 columns & 9.00 & 3.00 \\ 
  first 20 rows & 9.00 & 3.00 \\ 
   \hline
\end{tabular}
\caption{Comparison of results} 
\label{tab:por}
\end{table}


We can observe that the results from $IPC_1$ and $PC_1$ differ depending on the size of the data.
In particular for the sample of the first 20 columns the Information criteria look as follows

\begin{knitrout}
\definecolor{shadecolor}{rgb}{0.969, 0.969, 0.969}\color{fgcolor}\begin{figure}[H]

{\centering \includegraphics[width=\maxwidth]{figure/ex1_ic20-1} 

}

\caption[Information criteria for the first 20 columns]{Information criteria for the first 20 columns}\label{fig:ex1.ic20}
\end{figure}


\end{knitrout}






%%%%%%%%%%%%%%%%%%%%%%%%%%%%%%%%%%%%%%%%%%%%%%%%%%%%%%%%%%%%%%%%%%%%%%%%%%%%%%%%%%%%%%%%%%%%%%%%%%%%%%%%%%%%%%%%%%%%%%%%%%%%%%%%

\section{Exercise 2}

In this exercise we will be working with data representing electricity prices from the balancing market. Each row describes the day, whereas the column describes the hour.

\subsection{Part 1}
We transform the data into logarithms and calculate mean for each column. Then we subtract the mean from each column.

\begin{knitrout}
\definecolor{shadecolor}{rgb}{0.969, 0.969, 0.969}\color{fgcolor}\begin{kframe}
\begin{alltt}
\hlstd{Y} \hlkwb{<-} \hlkwd{as.matrix}\hlstd{(}\hlkwd{read_excel}\hlstd{(}\hlstr{'RB.xlsx'}\hlstd{,} \hlkwc{col_names} \hlstd{=} \hlnum{FALSE}\hlstd{))}
\hlstd{N} \hlkwb{<-} \hlkwd{ncol}\hlstd{(Y); T} \hlkwb{<-} \hlkwd{nrow}\hlstd{(Y)}
\hlstd{Y} \hlkwb{<-} \hlkwd{scale}\hlstd{(}\hlkwd{log}\hlstd{(Y),} \hlkwc{center} \hlstd{=} \hlnum{TRUE}\hlstd{,} \hlkwc{scale} \hlstd{=} \hlnum{FALSE}\hlstd{)}
\end{alltt}
\end{kframe}
\end{knitrout}

\begin{knitrout}
\definecolor{shadecolor}{rgb}{0.969, 0.969, 0.969}\color{fgcolor}

{\centering \includegraphics[width=\maxwidth]{figure/ex2_1_2-1} 

}



\end{knitrout}




\subsection{Part 2}
The plot shows the variability of the factors
\begin{knitrout}
\definecolor{shadecolor}{rgb}{0.969, 0.969, 0.969}\color{fgcolor}\begin{figure}[H]

{\centering \includegraphics[width=\maxwidth]{figure/ex2_2-1} 

}

\caption[Share of explained variance]{Share of explained variance}\label{fig:ex2.2}
\end{figure}


\end{knitrout}

We can observe that if we want to have the Factor model which explains at least $80\%$ of panel variability, we have to choose $K$ equal to

\begin{knitrout}
\definecolor{shadecolor}{rgb}{0.969, 0.969, 0.969}\color{fgcolor}\begin{kframe}
\begin{verbatim}
## [1] 4
\end{verbatim}
\end{kframe}
\end{knitrout}

Now we can calculate $\hat{F}$ and $\hat{\lambda}$ and check the following condition
\[ \frac{F'F}{T}=I. \]
We calculate $\frac{F'F}{T}$ and the result is as follows
\begin{knitrout}
\definecolor{shadecolor}{rgb}{0.969, 0.969, 0.969}\color{fgcolor}\begin{kframe}
\begin{verbatim}
##               [,1]          [,2]          [,3]          [,4]
## [1,]  1.000000e+00 -2.754477e-17  8.460181e-17  1.048388e-16
## [2,] -2.754477e-17  1.000000e+00 -3.147974e-17 -1.433453e-16
## [3,]  8.460181e-17 -3.147974e-17  1.000000e+00 -4.216037e-17
## [4,]  1.048388e-16 -1.433453e-16 -4.216037e-17  1.000000e+00
\end{verbatim}
\end{kframe}
\end{knitrout}
Taking into account the numerical errors we can assume that the above matrix is an identity matrix.




\subsection{Part 3}
We want to compute the information criteria with $K_\text{max}=8$. They suggest the following number of factors:
% latex table generated in R 3.2.3 by xtable 1.8-2 package
% Fri Dec  1 12:54:31 2017
\begin{table}[H]
\centering
\begin{tabular}{rr}
  \hline
  & Suggested no. of factors \\ 
  \hline
PC1 & 8.00 \\ 
  IPC1 & 8.00 \\ 
   \hline
\end{tabular}
\caption{Suggested number of factors} 
\label{tab:2.3.factors}
\end{table}


What is more, if we increase $K_\text{max}$ the Information criteria will return the following results
\begin{knitrout}
\definecolor{shadecolor}{rgb}{0.969, 0.969, 0.969}\color{fgcolor}\begin{figure}[H]

{\centering \includegraphics[width=\maxwidth]{figure/ex2_p3-1} 

}

\caption[Information criteria for increased $K_{max}$]{Information criteria for increased $K_{max}$}\label{fig:ex2.p3}
\end{figure}


\end{knitrout}
The bigger $K_\text{max}$ we take then the bigger $K$ is returned from Information criteria. So we concluded that Information criteria don't work in this case.



\subsection{Part 4}
Since Information criteria don't work for this data, we take $K$ calculated from share of explained variance. We are going to plot loadings of the first two factors. We change signs of values in $\lambda$ and $F$, so that values in $\lambda$ in 17th column are non-negative.
\begin{knitrout}
\definecolor{shadecolor}{rgb}{0.969, 0.969, 0.969}\color{fgcolor}\begin{figure}[H]

{\centering \includegraphics[width=\maxwidth]{figure/ex2_loadings-1} 

}

\caption[Factor loadings for K=4]{Factor loadings for K=4}\label{fig:ex2.loadings}
\end{figure}


\end{knitrout}
Factor 1 describes the noon peak and Factor 2 describes lower prices of electricity at night.


%%%%%%%%%%%%%%%%%%%%%%%%%%%%%%%%%%%%%%%%%%%%%%%%%%%%%%%%%%%%%%%%%%%%%%%%%%%%%%%%%%%%%%%%%%%%%%%%%%%%%%%%%%%%%%%%%%%%%%%%%%%%%%%%

\end{document}
