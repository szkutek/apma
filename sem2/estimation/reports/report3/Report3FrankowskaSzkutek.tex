\documentclass[12pt, a4paper]{article}\usepackage[]{graphicx}\usepackage[]{color}
%% maxwidth is the original width if it is less than linewidth
%% otherwise use linewidth (to make sure the graphics do not exceed the margin)
\makeatletter
\def\maxwidth{ %
  \ifdim\Gin@nat@width>\linewidth
    \linewidth
  \else
    \Gin@nat@width
  \fi
}
\makeatother

\definecolor{fgcolor}{rgb}{0.345, 0.345, 0.345}
\newcommand{\hlnum}[1]{\textcolor[rgb]{0.686,0.059,0.569}{#1}}%
\newcommand{\hlstr}[1]{\textcolor[rgb]{0.192,0.494,0.8}{#1}}%
\newcommand{\hlcom}[1]{\textcolor[rgb]{0.678,0.584,0.686}{\textit{#1}}}%
\newcommand{\hlopt}[1]{\textcolor[rgb]{0,0,0}{#1}}%
\newcommand{\hlstd}[1]{\textcolor[rgb]{0.345,0.345,0.345}{#1}}%
\newcommand{\hlkwa}[1]{\textcolor[rgb]{0.161,0.373,0.58}{\textbf{#1}}}%
\newcommand{\hlkwb}[1]{\textcolor[rgb]{0.69,0.353,0.396}{#1}}%
\newcommand{\hlkwc}[1]{\textcolor[rgb]{0.333,0.667,0.333}{#1}}%
\newcommand{\hlkwd}[1]{\textcolor[rgb]{0.737,0.353,0.396}{\textbf{#1}}}%
\let\hlipl\hlkwb

\usepackage{framed}
\makeatletter
\newenvironment{kframe}{%
 \def\at@end@of@kframe{}%
 \ifinner\ifhmode%
  \def\at@end@of@kframe{\end{minipage}}%
  \begin{minipage}{\columnwidth}%
 \fi\fi%
 \def\FrameCommand##1{\hskip\@totalleftmargin \hskip-\fboxsep
 \colorbox{shadecolor}{##1}\hskip-\fboxsep
     % There is no \\@totalrightmargin, so:
     \hskip-\linewidth \hskip-\@totalleftmargin \hskip\columnwidth}%
 \MakeFramed {\advance\hsize-\width
   \@totalleftmargin\z@ \linewidth\hsize
   \@setminipage}}%
 {\par\unskip\endMakeFramed%
 \at@end@of@kframe}
\makeatother

\definecolor{shadecolor}{rgb}{.97, .97, .97}
\definecolor{messagecolor}{rgb}{0, 0, 0}
\definecolor{warningcolor}{rgb}{1, 0, 1}
\definecolor{errorcolor}{rgb}{1, 0, 0}
\newenvironment{knitrout}{}{} % an empty environment to be redefined in TeX

\usepackage{alltt}

%%%%%%%%%%%%%%%%%%%%%%%%%%%%%%%%%%%%%%%%%%%%%%%%%%%%%%%%%%%%%%%%
% LaTeX packages
%\usepackage[OT4]{polski}
\usepackage[utf8]{inputenc}
\usepackage[top=2.5cm, bottom=2.5cm, left=2cm, right=2cm]{geometry}
\usepackage{graphicx}
\usepackage{amsmath}
\usepackage{float}
\usepackage[colorlinks=true, linkcolor=blue]{hyperref}


%%%%%%%%%%%%%%%%%%%%%%%%%%%%%%%%%%%%%%%%%%%%%%%%%%%%%%%%%%%%%%%%
% global settings


\IfFileExists{upquote.sty}{\usepackage{upquote}}{}
\begin{document}

%%%%%%%%%%%%%%%%%%%%%%%%%%%%%%%%%%%%%%%%%%%%%%%%%%%%%%%%%%%%%%%%
% title page
\title{Estimation theory -- Report 3}
\author{Marta Frankowska, 208581 \\ Agnieszka Szkutek, 208619}
\maketitle
\tableofcontents 


%%%%%%%%%%%%%%%%%%%%%%%%%%%%%%%%%%%%%%%%%%%%%%%%%%%%%%%%%%%%%%%%
\section{Model}
In both exercises we will be using the Factor model
\[ Y_{T\times N} = F_{T\times K} \cdot \lambda_{K\times N} + e_{T\times N},\]
where 
\begin{itemize}
  \item $Y_{T\times N}$ panel of observations
  \item $F_{T\times K}$ matrix of common (latent) factors
  \item $\lambda_{K\times N}$ matrix of loadings
  \item $e_{T\times N}$ panel of specific components
\end{itemize}

To calculate $F$ and $\lambda$ we use the following formulas:
\[ \hat{F} = \sqrt{T} V_{1:K} \quad\text{and}\quad \hat{\lambda} = \frac{\hat{F}' Y}{T},\]
where $V_{1:K}$ are eigen vectors of $YY'$ corresponding to the $K$ largest eigenvalues.


\subsection{Selecting optimal number of factors}
Notation:
\begin{itemize}
  \item $K = 1,2,\dots,K_\text{max}$ - the number of factors
  \item $e^{(K)}$ - the individual components for $K$ factors
  \item $V(K) = \frac{1}{NT}\sum_{t=1}^{T} \sum_{i=1}^{N} \left( e^{(K)}_{t i} \right)^2$
  \item $\hat{\sigma}^2 = V(K_\text{max})$
\end{itemize}

Information criteria:
\begin{itemize}
  \item $PC_1(K) = V(K) + K \hat{\sigma}^2 \frac{N+T}{NT} \ln{\frac{NT}{N+T}}$
  \item $IPC_1(K) = \log{V(K)} + K\frac{N+T}{NT} \ln{\frac{NT}{N+T}}$
\end{itemize}

Algorithm:
\begin{enumerate}
	\item Set $K_{max}$
	\item Compute $IC(K)$ for $K = 1...K_{max}$
	\item Choose $\hat{K}$ such that $IC(\hat{K}) = \min_{1\le K\le K_{max}} IC(K)$
\end{enumerate}


\begin{knitrout}
\definecolor{shadecolor}{rgb}{0.969, 0.969, 0.969}\color{fgcolor}\begin{kframe}
\begin{alltt}
\hlstd{factor.model.est} \hlkwb{<-} \hlkwa{function}\hlstd{(}\hlkwc{Y}\hlstd{,}\hlkwc{K_max}\hlstd{)} \hlcom{#function returning which K we should choose}
  \hlstd{\{}
  \hlstd{T} \hlkwb{<-} \hlkwd{nrow}\hlstd{(Y)}
  \hlstd{N} \hlkwb{<-} \hlkwd{ncol}\hlstd{(Y)}

  \hlstd{eigen.decomp} \hlkwb{<-} \hlkwd{eigen}\hlstd{(Y} \hlopt \hlkwd{t}\hlstd{(Y))} \hlcom{#  }
  \hlstd{eigen.values} \hlkwb{<-} \hlstd{eigen.decomp}\hlopt{$}\hlstd{values}
  \hlstd{eigen.vectors} \hlkwb{<-} \hlstd{eigen.decomp}\hlopt{$}\hlstd{vector}

  \hlcom{# we calculate F, lambda and e for K_max}
  \hlstd{F} \hlkwb{<-} \hlkwd{sqrt}\hlstd{(T)}\hlopt{*}\hlstd{eigen.vectors[,}\hlnum{1}\hlopt{:}\hlstd{K_max]}
  \hlstd{lambda} \hlkwb{<-} \hlkwd{t}\hlstd{(F)}\hlopt\hlstd{Y}\hlopt{/}\hlstd{T}
  \hlstd{e} \hlkwb{<-} \hlstd{Y} \hlopt{-} \hlstd{F}\hlopt\hlstd{lambda}
  \hlstd{sigma2.hat} \hlkwb{<-} \hlkwd{sum}\hlstd{(e}\hlopt{^}\hlnum{2}\hlstd{)}\hlopt{/}\hlstd{(N}\hlopt{*}\hlstd{T)}

  \hlstd{PC1} \hlkwb{<-} \hlnum{1}\hlopt{:}\hlstd{K_max}
  \hlstd{IPC1} \hlkwb{<-} \hlnum{1}\hlopt{:}\hlstd{K_max}
  \hlkwa{for} \hlstd{(K} \hlkwa{in} \hlnum{1}\hlopt{:}\hlstd{K_max)}
  \hlstd{\{}
   \hlcom{# we calculate F, lambda and e for K}
   \hlstd{F} \hlkwb{<-} \hlkwd{sqrt}\hlstd{(T)}\hlopt{*}\hlstd{eigen.vectors[,}\hlnum{1}\hlopt{:}\hlstd{K]}
   \hlstd{lambda} \hlkwb{<-} \hlkwd{t}\hlstd{(F)}\hlopt\hlstd{Y}\hlopt{/}\hlstd{T}
   \hlstd{e} \hlkwb{<-} \hlstd{Y} \hlopt{-} \hlstd{F}\hlopt\hlstd{lambda}
   \hlstd{V} \hlkwb{<-} \hlkwd{sum}\hlstd{(e}\hlopt{^}\hlnum{2}\hlstd{)}\hlopt{/}\hlstd{(N}\hlopt{*}\hlstd{T)}
   \hlcom{# we calculate PC1 and IPC1 for K}
   \hlstd{PC1[K]} \hlkwb{<-} \hlstd{V} \hlopt{+} \hlstd{K}\hlopt{*}\hlstd{sigma2.hat}\hlopt{*}\hlstd{((N}\hlopt{+}\hlstd{T)}\hlopt{/}\hlstd{(N}\hlopt{*}\hlstd{T))}\hlopt{*}\hlkwd{log}\hlstd{(N}\hlopt{*}\hlstd{T}\hlopt{/}\hlstd{(N}\hlopt{+}\hlstd{T))}
   \hlstd{IPC1[K]} \hlkwb{<-} \hlkwd{log}\hlstd{(V)} \hlopt{+} \hlstd{K}\hlopt{*}\hlstd{((N}\hlopt{+}\hlstd{T)}\hlopt{/}\hlstd{(N}\hlopt{*}\hlstd{T))}\hlopt{*}\hlkwd{log}\hlstd{(N}\hlopt{*}\hlstd{T}\hlopt{/}\hlstd{(N}\hlopt{+}\hlstd{T))}
  \hlstd{\}}
  \hlcom{#we choose minimum for PC1 and IPC1}
  \hlstd{min_PC1} \hlkwb{<-} \hlkwd{min}\hlstd{(PC1)}
  \hlstd{min_IPC1} \hlkwb{<-} \hlkwd{min}\hlstd{(IPC1)}
  \hlcom{# and looking for corresponding K}
  \hlstd{PC1_K} \hlkwb{<-} \hlkwd{which}\hlstd{(PC1} \hlopt{==} \hlstd{min_PC1)}
  \hlstd{IPC1_K} \hlkwb{<-} \hlkwd{which}\hlstd{(IPC1} \hlopt{==} \hlstd{min_IPC1)}
  \hlkwd{return} \hlstd{(}\hlkwd{list}\hlstd{(PC1_K,IPC1_K))}
\hlstd{\}}
\end{alltt}
\end{kframe}
\end{knitrout}

\section{Exercise 1}
We will be using data from file \textit{dataLab3.xlsx}, where $Y$ size is $T\times N = 100\times 100$.
To calculate the number of factors $K$ we will use function \textit{factor.model.est}.


\subsection{Part 1}
First, we calculate the number of factors for the whole sample. The function returns $\hat{K}=3$ for both $PC_1$ and $IPC_1$.
Share of explained variance
\[ \frac{\sum_{i=1}^{K} \gamma_i }{\sum_{i=1}^{T} \gamma_i}, \]
where $\gamma_i$ are the eigenvalues of $Y Y'$. It can be used to choose the number of factors.

The plot shows the variability of the factors
\begin{knitrout}
\definecolor{shadecolor}{rgb}{0.969, 0.969, 0.969}\color{fgcolor}\begin{figure}[H]

{\centering \includegraphics[width=\maxwidth]{figure/ex1_1-1} 

}

\caption[Share of explained variance depending on K]{Share of explained variance depending on K}\label{fig:ex1.1}
\end{figure}


\end{knitrout}
We can observe that for $K$ we calculated from Information Criteria, the share of explained variance is more than 0.8.


Now we will compare estimated number of factors for the whoe sample, the first 20 columns and the first 20 rows.
% latex table generated in R 3.2.3 by xtable 1.8-2 package
% Wed Nov 22 15:08:34 2017
\begin{table}[H]
\centering
\begin{tabular}{rrr}
  \hline
  & PC1 & IPC1 \\ 
  \hline
whole sample & 3.00 & 3.00 \\ 
  first 20 columns & 9.00 & 3.00 \\ 
  first 20 rows & 9.00 & 3.00 \\ 
   \hline
\end{tabular}
\caption{Comparison of results} 
\label{tab:por}
\end{table}


We can observe that ...



%%%%%%%%%%%%%%%%%%%%%%%%%%%%%%%%%%%%%%%%%%%%%%%%%%%%%%%%%%%%%%%%%%%%%%%%%%%%%%%%%%%%%%%%%%%%%%%%%%%%%%%%%%%%%%%%%%%%%%%%%%%%%%%%

\section{Exercise 2}

In this exercise we will be working with data representing electricity prices from the balancing market. Each row describes the day, whereas the column describes the hour.

\subsection{Part 1}
We transform the data into logarithms and calculate mean for each column. Then we subtract the mean from each column.

\begin{knitrout}
\definecolor{shadecolor}{rgb}{0.969, 0.969, 0.969}\color{fgcolor}\begin{kframe}
\begin{alltt}
\hlstd{data2} \hlkwb{<-} \hlkwd{read_excel}\hlstd{(}\hlstr{'RB.xlsx'}\hlstd{,} \hlkwc{col_names} \hlstd{=} \hlnum{FALSE}\hlstd{)}
\hlstd{Y} \hlkwb{<-} \hlkwd{as.matrix}\hlstd{(data2)}
\hlcom{#1}
\hlstd{N} \hlkwb{<-} \hlkwd{ncol}\hlstd{(Y)}
\hlstd{T} \hlkwb{<-} \hlkwd{nrow}\hlstd{(Y)}
\hlstd{log_Y} \hlkwb{<-} \hlkwd{log}\hlstd{(Y)}
\hlstd{new_Y} \hlkwb{<-} \hlkwd{sweep}\hlstd{(log_Y,} \hlnum{2}\hlstd{,} \hlkwd{colMeans}\hlstd{(log_Y))}
\end{alltt}
\end{kframe}
\end{knitrout}


\subsection{Part 2}
The plot shows the variability of the factors
\begin{knitrout}
\definecolor{shadecolor}{rgb}{0.969, 0.969, 0.969}\color{fgcolor}\begin{figure}[H]

{\centering \includegraphics[width=\maxwidth]{figure/ex2_2-1} 

}

\caption[Share of explained variance]{Share of explained variance}\label{fig:ex2.2}
\end{figure}


\end{knitrout}

We can observe that if we want to have the Factor model which explains at least $80\%$ of panel variability, we have to choose $K=4$.


\subsection{Part 3}
We want to compute the information criteria with $K_\text{max}=8$. They suggest the following number of factors:
% latex table generated in R 3.2.3 by xtable 1.8-2 package
% Wed Nov 22 15:08:37 2017
\begin{table}[H]
\centering
\begin{tabular}{rr}
  \hline
  & Suggested no. of factors \\ 
  \hline
PC1 & 8.00 \\ 
  IPC1 & 8.00 \\ 
   \hline
\end{tabular}
\caption{Suggested number of factors} 
\label{tab:2.3.factors}
\end{table}








%%%%%%%%%%%%%%%%%%%%%%%%%%%%%%%%%%%%%%%%%%%%%%%%%%%%%%%%%%%%%%%%%%%%%%%%%%%%%%%%%%%%%%%%%%%%%%%%%%%%%%%%%%%%%%%%%%%%%%%%%%%%%%%%

\end{document}
